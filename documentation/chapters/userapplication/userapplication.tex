\section{BenutzerInnen-Anwendung}

Bei der BenutzerInnen-Anwendung handelt es sich um die Ansteuerung eines Moving Heads mittels DMX Protokoll.  

\subsection{Grundlegender Aufbau des DMX Protokolls}
Es gibt mehrere verschiedene Spezifikationen für das DMX Protokoll. Im folgenden wird eine Spezifikation erläutert und anschließend zu Vergleichszwecken verwendet. Abbildung \ref{fig:DMX-512-Protocol} dient zur Veranschaulichung des DMX-512 Protokolls. 

\begin{figure}[H]
	\includegraphics[scale=0.60]{chapters/userapplication/figures/todo}
	\caption{Memory Map des Betriebssystems}
	\label{fig:DMX-512-Protocol}
\end{figure}

\begin{table}[H]
\begin{tabular}{ l | l | l | l | l | l }
  \textbf{Nummer} & \textbf{Signalname} & \textbf{Min.} & \textbf{Typ.} & \textbf{Max.} & \textbf{Einheit} \\ 
  \hline
  1 & Reset & 88 & 88 & - & $\mu$ s \\
 \end{tabular}
 \caption{Eigenschaften des DMX-512-Protokolls}
 \label{table:DMX-512-Protocol}
\end{table}


\pagebreak 