\section{Sicherheitsaspekte}

Hinsichtlich der Sicherheit werden an das Betriebssystems die Anforderungen der strikten Trennung von Prozessadressräumen und der Trennung von privilegierten und nichtprivilegierten Modi gestellt. Mögliche Sicherheitsrisiken und deren Vermeidung werden nachfolgend beschrieben.

 

\subsection{Sicherheitsrisiken}

Aufschluss über mögliche Sicherheitsrisiken ergibt eine nähere Betrachung des Speichermodells in Abbildung \ref{fig:MemoryMap}. Wie die Abbildung zeigt, beginnt im virtuellen Speicher der Adressbereich für Prozesse ab der Adresse 0x00000000. Gleichzeitig befindet sich die Startadresse für die ROM exception vector table an der Adresse 0x00020000. Durch diese Gegebenheiten bestehen zwei grundsätzliche Sicherheitsrisiken:

\begin{enumerate}
	\item Nulladressenproblem: Adresse 0x00000000 ist im Regelfall reserviert für Nullpointer
	\item Anfälligkeit für Hacking durch unsaubere Adressraumtrennung: ROM exception vector table muss bei dieser Konstellation in den page tables für Prozesse direkt, d.h. eins zu eins, gemappt sein
\end{enumerate}

Letzteres Sicherheitsrisiko bietet Hackern die Möglichkeit, durch sukzessives Erhöhen der angesprochenen Adresse vom User Mode in den System Mode zu gelangen. Damit wäre eine Hackeranwendung in der Lage, mit voller Befugnis auf die Hardware zuzugreifen und Programmteile des Kernels auszuführen.

Die Lösung für diese beiden Sicherheitsrisiken wird im Folgenden vorgestellt.

\subsection{Vermeidung des Nulladressenproblems}

Die Lösung des Nulladressenproblems kann mit relativ wenig Aufwand erreicht werden. In der Memory Region für den Prozessadressbereich wird die erste Page für alle Prozesse bereits beim Erstellen reserviert. Dadurch wird vermieden, dass bei einer Speicherallokation die Nulladresse oder eine non-aligned Adresse ausgegeben wird. Zusätzlich werden im DABT-Handler, der für die Einlagerung von Adressen von page frames in die page tables von Prozessen zuständig ist, diese nun nicht erlaubten Adressen abgefangen. Tritt aus welchem Grund auch immer eine Adresse aus dem Adressbereich der ersten page im DABT-Handler auf, wird der entsprechende Prozess gekillt und der nächste bereite Prozess zur Ausführung gebracht.

\subsection{Implementierung der Hivecs}

\pagebreak 