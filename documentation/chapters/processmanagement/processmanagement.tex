\section{Prozessverwaltung}
Als Prozessverwaltung wird hauptsächlich das Verwalten von Prozessen durch das Betriebssystem verstanden (Vgl. http://www.lowlevel.eu/wiki/Prozessverwaltung). Jeder Prozess besitzt eine eindeutige Identifikation (PID), durch welche dieser vom System angesprochen werden kann. 

\subsection{Prozesszustände}
Jeder Prozess besitzt zu einem bestimmten Zeitpunkt einen fix definierten Zustand, d.h. es können keine Inkonsistenzen auftreten. Abbildung \ref{fig:Process-states} zeigt die verschiedenen Zustände eines Prozesses sowie die jeweilig erlaubten Übergänge zu einem anderen Zustand auf.

\begin{figure}[H]
	\includegraphics[scale=0.60]{chapters/processmanagement/figures/todo}
	\caption{Erlaubte Prozesszustände und Prozessübergänge}
	\label{fig:Process-states}
\end{figure}

Im folgenden wird eine detaillierte Erklärung zu den einzelnen Zuständen aus Abbildung \ref{fig:Process-states} gegeben.

\begin{description}
	\item[Free] \hfill \\
	XXX
	
	\item[Running] \hfill \\
	XXX
	
	\item[Blocked] \hfill \\
	XXX
	
	\item[Finished] \hfill \\
	XXX	
\end{description}

Es gibt unterschiedliche Zustandsübergänge, welche im Betriebssystem erlaubt sind. Tabelle \ref{table:State-transition} stellt die verschiedenen Übergänge mit einem dazu passenden Beispiel dar.

\begin{table}[H]
\begin{tabular}{p{2.5cm} | p{2.5cm} | p{8cm}}
  \textbf{Ausgangszustand} & \textbf{Nächster Zustand} & \textbf{Beispiel} 
  \\ \hline
  Ready & Running & XXX \\
  Running & Blocked & XXX \\
  Running & Finised & XXX \\
  XXX & XXX & XXX \\
  
 \end{tabular}
 \caption{Erlaubte Zustandsübergänge mit Beispiel}
 \label{table:State-transition}
\end{table}

\pagebreak 